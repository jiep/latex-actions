\documentclass{article}

\usepackage{amsmath}
\usepackage{fontspec}
\usepackage{unicode-math}

\setmainfont[ Path = fonts/, 
UprightFont = *-regular,
BoldFont = *-bold,
BoldItalicFont = *-bolditalic
]{texgyrepagella}

\setmathfont[Path = fonts/]{texgyrepagella-math.otf}

\title{\LaTeX}
\author{José Ignacio Escribano}

\begin{document}
  \maketitle
  \LaTeX{} is a document preparation system for
  the \TeX{} typesetting program. It offers
  programmable desktop publishing features and
  extensive facilities for automating most
  aspects of typesetting and desktop publishing,
  including numbering and  cross-referencing,
  tables and figures, page layout,
  bibliographies, and much more. \LaTeX{} was
  originally written in 1984 by Leslie Lamport
  and has become the  dominant method for using
  \TeX; few people write in plain \TeX{} anymore.
  The current version is \LaTeXe.

  % This is a comment, not shown in final output.
  % The following shows typesetting  power of LaTeX:
  \begin{align}
    E_0 &= mc^2 \\
    E &= \frac{mc^2}{\sqrt{1-\frac{v^2}{c^2}}}
  \end{align}

  The most famous book to learn \TeX{} and \LaTeX{}
  are~\cite{texbook} and~\cite{latex}, respectively.

  \bibliographystyle{unsrt}
  \bibliography{bibliography}
\end{document}
